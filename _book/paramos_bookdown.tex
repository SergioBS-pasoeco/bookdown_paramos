% Options for packages loaded elsewhere
\PassOptionsToPackage{unicode}{hyperref}
\PassOptionsToPackage{hyphens}{url}
%
\documentclass[
]{book}
\usepackage{lmodern}
\usepackage{amssymb,amsmath}
\usepackage{ifxetex,ifluatex}
\ifnum 0\ifxetex 1\fi\ifluatex 1\fi=0 % if pdftex
  \usepackage[T1]{fontenc}
  \usepackage[utf8]{inputenc}
  \usepackage{textcomp} % provide euro and other symbols
\else % if luatex or xetex
  \usepackage{unicode-math}
  \defaultfontfeatures{Scale=MatchLowercase}
  \defaultfontfeatures[\rmfamily]{Ligatures=TeX,Scale=1}
\fi
% Use upquote if available, for straight quotes in verbatim environments
\IfFileExists{upquote.sty}{\usepackage{upquote}}{}
\IfFileExists{microtype.sty}{% use microtype if available
  \usepackage[]{microtype}
  \UseMicrotypeSet[protrusion]{basicmath} % disable protrusion for tt fonts
}{}
\makeatletter
\@ifundefined{KOMAClassName}{% if non-KOMA class
  \IfFileExists{parskip.sty}{%
    \usepackage{parskip}
  }{% else
    \setlength{\parindent}{0pt}
    \setlength{\parskip}{6pt plus 2pt minus 1pt}}
}{% if KOMA class
  \KOMAoptions{parskip=half}}
\makeatother
\usepackage{xcolor}
\IfFileExists{xurl.sty}{\usepackage{xurl}}{} % add URL line breaks if available
\IfFileExists{bookmark.sty}{\usepackage{bookmark}}{\usepackage{hyperref}}
\hypersetup{
  pdftitle={Ecología de la vegetación de los páramos de Colombia usando R},
  pdfauthor={Björn Reu, Zarith Villamizar \& Sergio Bolívar},
  hidelinks,
  pdfcreator={LaTeX via pandoc}}
\urlstyle{same} % disable monospaced font for URLs
\usepackage{longtable,booktabs}
% Correct order of tables after \paragraph or \subparagraph
\usepackage{etoolbox}
\makeatletter
\patchcmd\longtable{\par}{\if@noskipsec\mbox{}\fi\par}{}{}
\makeatother
% Allow footnotes in longtable head/foot
\IfFileExists{footnotehyper.sty}{\usepackage{footnotehyper}}{\usepackage{footnote}}
\makesavenoteenv{longtable}
\usepackage{graphicx}
\makeatletter
\def\maxwidth{\ifdim\Gin@nat@width>\linewidth\linewidth\else\Gin@nat@width\fi}
\def\maxheight{\ifdim\Gin@nat@height>\textheight\textheight\else\Gin@nat@height\fi}
\makeatother
% Scale images if necessary, so that they will not overflow the page
% margins by default, and it is still possible to overwrite the defaults
% using explicit options in \includegraphics[width, height, ...]{}
\setkeys{Gin}{width=\maxwidth,height=\maxheight,keepaspectratio}
% Set default figure placement to htbp
\makeatletter
\def\fps@figure{htbp}
\makeatother
\setlength{\emergencystretch}{3em} % prevent overfull lines
\providecommand{\tightlist}{%
  \setlength{\itemsep}{0pt}\setlength{\parskip}{0pt}}
\setcounter{secnumdepth}{5}
\usepackage{booktabs}
\usepackage{amsthm}
\makeatletter
\def\thm@space@setup{%
  \thm@preskip=8pt plus 2pt minus 4pt
  \thm@postskip=\thm@preskip
}
\makeatother
\AtBeginDocument{\renewcommand{\chaptername}{Capítulo}}
\ifluatex
  \usepackage{selnolig}  % disable illegal ligatures
\fi
\usepackage[]{natbib}
\bibliographystyle{apalike}

\title{Ecología de la vegetación de los páramos de Colombia usando R}
\author{Björn Reu, Zarith Villamizar \& Sergio Bolívar}
\date{2020-08-07}

\begin{document}
\maketitle

{
\setcounter{tocdepth}{1}
\tableofcontents
}
\hypertarget{introducciuxf3n}{%
\chapter*{Introducción}\label{introducciuxf3n}}
\addcontentsline{toc}{chapter}{Introducción}

Este libro provee la información necesaria para manipular, depurar y analizar la información correspondiente a la composición de la vegetación en \textbf{los páramos de Colombia}. Utilizaremos para esto, datos de ocurrencias del \textbf{GBIF} y cada capitulo te llevará por el paso a paso a seguir para analizar este tipo de datos. Aprenderemos a estimar diversidad verdadera o números de Hill con el paquete \textbf{iNEXT}, realizar pruebas de hipótesis, regresiones lineales, análisis multivariados y mucho más con R.

\begin{figure}
\centering
\includegraphics[width=0.5\textwidth,height=\textheight]{map_par_V5.png}
\caption{Fig. 1. Complejos de Páramo en el Neotrópico}
\end{figure}

\begin{quote}
\textbf{\emph{Sugerencia}}

Para un mejor aprovechamiento de este libro y su aplicación usando R, es pertinente tener conocimientos básicos sobre la sintaxis y la estructura de datos que vamos a utilizar. Les recomendamos revisar nuestro curso básico de R en el siguiente enlace\ldots{}
\end{quote}

\hypertarget{los-datos}{%
\chapter{Los datos}\label{los-datos}}

\hypertarget{ocurrencias-de-especies-desde-gbif}{%
\section{Ocurrencias de especies desde GBIF}\label{ocurrencias-de-especies-desde-gbif}}

\begin{quote}
\texttt{“Aquí\ aprenderás\ a\ descargar\ ocurrencias\ de\ especies\ de\ Páramo\ con\ RGbif”}

¿Debes descargar una gran cantidad de datos de ocurrencias de especies y quieres hacerlo de una manera eficiente?
\end{quote}

\hypertarget{correciuxf3n-de-los-sinuxf3nimos-utilizando-tnrs}{%
\section{Correción de los sinónimos utilizando TNRS}\label{correciuxf3n-de-los-sinuxf3nimos-utilizando-tnrs}}

\begin{quote}
\texttt{“Aquí\ aprenderás\ a\ eliminar\ sinónimos\ utilizando\ Taxize”}

\ldots En ocasiones los nombres de nuestras especies son sinónimos que aún no han sido actualizados o simplemente están mal escrotos ¿Qué podemos hacer?
\end{quote}

\hypertarget{mapeo-de-los-puxe1ramos-de-colombia}{%
\section{Mapeo de los páramos de Colombia}\label{mapeo-de-los-puxe1ramos-de-colombia}}

\begin{quote}
\texttt{“Aquí\ aprenderás\ a\ realizar\ un\ mapa\ de\ los\ Páramos\ en\ R\ utilizando\ Raster”}
\end{quote}

\hypertarget{datos-ambientales-de-worldclim-y-otras-fuentes}{%
\section{Datos ambientales de WorldClim y otras fuentes}\label{datos-ambientales-de-worldclim-y-otras-fuentes}}

\begin{quote}
\texttt{“Aquí\ aprenderás\ a\ descargar\ datos\ climáticos\ de\ WorldClim”}
\end{quote}

\hypertarget{extracciuxf3n-de-caracteruxedsticas-ambientales-utilizando-el-paquete-raster}{%
\section{Extracción de características ambientales utilizando el paquete ``raster''}\label{extracciuxf3n-de-caracteruxedsticas-ambientales-utilizando-el-paquete-raster}}

\begin{quote}
\texttt{“Aquí\ aprenderás\ a\ extraer\ información\ espacial\ de\ los\ páramos\ y\ combinarla\ datos\ de\ diferentes\ fuentes”}
\end{quote}

\hypertarget{exploraciuxf3n-de-los-datos}{%
\chapter{Exploración de los datos}\label{exploraciuxf3n-de-los-datos}}

\hypertarget{explorando-los-datos-de-las-especies}{%
\section{Explorando los datos de las especies}\label{explorando-los-datos-de-las-especies}}

\begin{quote}
\texttt{“Aquí\ aprendes\ limpiar\ y\ filtrar\ los\ ocurrencias\ de\ las\ especies\ y\ calcular\ índices\ simples\ para\ entenderlos\ mejor”}
\end{quote}

\hypertarget{depuraciuxf3n-de-los-sitios-con-pocas-observaciones-especies}{%
\section{Depuración de los sitios con pocas observaciones especies}\label{depuraciuxf3n-de-los-sitios-con-pocas-observaciones-especies}}

\begin{quote}
\texttt{“Aquí\ aprendes\ a\ eliminar\ observaciones\ no\ representativos”}
\end{quote}

\hypertarget{especies-uxfanicas}{%
\section{Especies únicas}\label{especies-uxfanicas}}

\begin{quote}
\texttt{“Aquí\ aprendes\ a\ identificar\ las\ especies\ únicas\ dentro\ de\ tus\ datos”}
\end{quote}

\hypertarget{explorando-datos-medioambientales}{%
\section{Explorando datos medioambientales}\label{explorando-datos-medioambientales}}

\begin{quote}
\texttt{“Aquí\ aprendes\ a\ identificar\ los\ Páramos\ más\ calientes,\ más\ húmedos,\ más\ secos,\ más\ pequeños,\ \ldots{}.”}
\end{quote}

\hypertarget{gradientes-medioambientales-con-pca}{%
\section{Gradientes medioambientales con PCA}\label{gradientes-medioambientales-con-pca}}

\begin{quote}
\texttt{“Aquí\ aprendes\ sobre\ el\ Análisis\ de\ Componentes\ Principales\ y\ a\ extraer\ gradientes\ ambientales”}
\end{quote}

\hypertarget{biodiversidad}{%
\chapter{Biodiversidad}\label{biodiversidad}}

\hypertarget{riqueza-de-especies-vs.-intensidad-de-muestreo}{%
\section{Riqueza de especies vs.~intensidad de muestreo}\label{riqueza-de-especies-vs.-intensidad-de-muestreo}}

\begin{quote}
\texttt{“Aquí\ aprenderás\ sobre\ la\ representatividad\ de\ los\ muestreos\ y\ la\ riqueza\ verdadera”}
\end{quote}

\hypertarget{probando-hipuxf3tesis-de-biodiversidad}{%
\section{Probando hipótesis de biodiversidad}\label{probando-hipuxf3tesis-de-biodiversidad}}

\begin{quote}
\texttt{“Aquí\ aprenderás\ poner\ a\ prueba\ diferentes\ hipótesis\ sobre\ porqué\ varia\ la\ biodiversidad\ entre\ lugares”}
\end{quote}

\hypertarget{composiciuxf3n-floruxedstica}{%
\chapter{Composición florística}\label{composiciuxf3n-floruxedstica}}

\hypertarget{agrupamientos-basados-en-la-composiciuxf3n-de-especies}{%
\section{Agrupamientos basados en la composición de especies}\label{agrupamientos-basados-en-la-composiciuxf3n-de-especies}}

\begin{quote}
\texttt{“Aquí\ aprenderás\ a\ reconocer\ sitios\ similares\ entre\ sí,\ de\ acuerdo\ a\ las\ especies\ que\ contienen”}
\end{quote}

\hypertarget{gradientes-floruxedsticos}{%
\section{Gradientes florísticos}\label{gradientes-floruxedsticos}}

\begin{quote}
\texttt{“Aquí\ aprenderás\ a\ realizar\ análisis\ de\ ordenamiento,\ utilizando\ datos\ binario\ (PCoA\ y\ NMDS)”}
\end{quote}

\hypertarget{determinantes-de-la-diversidad-floruxedstica}{%
\section{Determinantes de la diversidad florística}\label{determinantes-de-la-diversidad-floruxedstica}}

\begin{quote}
\texttt{“Aquí\ aprenderás\ cuales\ variables\ ambientales\ son\ importantes\ para\ la\ composición\ de\ las\ plantas\ presentes\ en\ cada\ páramos”}
\end{quote}

\hypertarget{identificando-especies-indicadoras}{%
\section{Identificando especies indicadoras}\label{identificando-especies-indicadoras}}

\begin{quote}
\texttt{“Aquí\ aprenderás\ a\ identificar\ las\ especies\ clave\ para\ entender\ el\ gradiente\ florístico\ en\ los\ ensamblajes\ analizados”}
\end{quote}

  \bibliography{book.bib,packages.bib}

\end{document}
